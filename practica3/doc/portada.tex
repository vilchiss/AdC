\documentclass[12pt]{article}

\usepackage[a4paper, top=2cm, bottom=2cm, left=2.2cm, right=2.2cm]{geometry}
\usepackage[T1]{fontenc}
\usepackage[utf8x]{inputenc}
\usepackage{lmodern}
\usepackage[spanish]{babel}
\usepackage{graphicx}
\usepackage{float}
\usepackage{multirow}
\usepackage{listings}

\author{LIDSOL Team :v}
\title{Pr\'actica 3}
\date{10 Abril del 2018}
\def\universidad{Universidad Nacional Aut\'onoma de M \'exico}
\def\facultad{Facultad de Ingenier\'ia}
\def\materia{Arquitectura de Computadoras}
\def\grupo{Grupo: 3}
\def\profesor{M.I. Jos\'e Luis Cruz Mora}
\def\luis{Oropeza Vilchis Luis Alberto}
\def\diego{Barriga Mart\'inez Diego Alberto}
\def\emilio{Cabrera L\'opez Oscar Emilio}
\makeatletter

\begin{document}
\pagenumbering{gobble}

% Escudos
\begin{figure}
	\begin{flushleft}
		\includegraphics[width=3cm]{unam}\hfill 				\includegraphics[width=3cm]{fi2}
	\end{flushleft}
\end{figure}

\begin{center}
	\Huge \universidad \\
	\hfill \\
	\facultad \\
	\hfill \\
    \materia\par
    \hfill \\
    \@title\par
    \hfill \\
    \profesor\par
    \hfill \\
    Integrantes:
    \hfill \\
    \hfill \\
    \diego \\
    \emilio \\
    \luis \\
    \hfill \\
    \grupo \\
    \hfill \\
    \@date \\
\end{center}

\newpage
\pagenumbering{arabic}

\section{Direccionamiento Entrada-Estado}

\begin{figure}[H]
	\centering
	\includegraphics[width=0.8\textwidth]{carta_asm}
	\caption{Carta ASM modificada}
\end{figure}

\medskip
\noindent \textbf{N\'umero de estados:} 14, por lo que se utilizan 4 bits para representarlos\\
\textbf{Entradas:} $S_{i}$ y $S_{d}$\\
\textbf{Salidas:} Adelante (Ad), Atr\'as (At), Giro Izquierda (GI), Giro Derecha (GD)\\
\textbf{Prueba}:\\

\begin{center}
	\begin{tabular}{|c|c|}
		\hline
		 Entrada & C\'odigo\\
		\hline 
		$S_{i}$ & 0 0 \\ 
		\hline 
		$S_{d}$ & 0 1 \\ 
		\hline 
		$\uparrow Aux$ & 1 1 \\ 
		\hline 
	\end{tabular}\\
\end{center}

\noindent \textbf{ Contenido de la memoria:}\\
\begin{center}
	\begin{tabular}{|c|c|c|c|c|c|c|}
		\hline 
		\multicolumn{2}{|c|}{ \multirow{2}{*}{Estado}} & \multirow{2}{*}{Prueba} & \multirow{2}{*}{LV} & \multirow{2}{*}{LF} & SV & SF \\ 
		\cline{6-7}
		 \multicolumn{2}{|c|}{}& & & &Ad At GI GD & Ad At GI GD \\ 
		\hline 
		0   & 0 0 0 0 & 0 0 & 0 0 0 1 & 0 0 1 0 & 0 0 0 0 & 0 0 0 0 \\ 
		\hline                                                     
		1   & 0 0 0 1 & 0 1 & 0 1 0 1 & 0 0 1 1 & 0 0 0 0 & 0 0 0 0 \\ 
		\hline                                                     
		2   & 0 0 1 0 & 0 1 & 1 1 0 0 & 0 0 0 0 & 0 0 0 0 & 1 0 0 0 \\ 
		\hline                                                     
		3   & 0 0 1 1 & 1 1 & 0 1 0 0 & 0 1 0 0 & 0 1 0 0 & 0 1 0 0 \\ 
		\hline                                                     
		4   & 0 1 0 0 & 1 1 & 0 0 0 0 & 0 0 0 0 & 0 0 0 1 & 0 0 0 1 \\ 
		\hline                                                     
		5   & 0 1 0 1 & 1 1 & 0 1 1 0 & 0 1 1 0 & 0 1 0 0 & 0 1 0 0 \\ 
		\hline                                                     
		6   & 0 1 1 0 & 1 1 & 0 1 1 1 & 0 1 1 1 & 0 0 1 0 & 0 0 1 0 \\ 
		\hline                                                     
		7   & 0 1 1 1 & 1 1 & 1 0 0 0 & 1 0 0 0 & 0 0 1 0 & 0 0 1 0 \\ 
		\hline                                                     
		8   & 1 0 0 0 & 1 1 & 1 0 0 1 & 1 0 0 1 & 1 0 0 0 & 1 0 0 0 \\ 
		\hline                                                     
		9   & 1 0 0 1 & 1 1 & 1 0 1 0 & 1 0 1 0 & 1 0 0 0 & 1 0 0 0 \\ 
		\hline                                                     
		10  & 1 0 1 0 & 1 1 & 1 0 1 1 & 1 0 1 1 & 0 0 0 1 & 0 0 0 1 \\ 
		\hline                                                     
		11  & 1 0 1 1 & 1 1 & 0 0 0 0 & 0 0 0 0 & 0 0 0 1 & 0 0 0 1 \\ 
		\hline                                                     
		12  & 1 1 0 0 & 1 1 & 1 1 0 1 & 1 1 0 1 & 0 1 0 0 & 0 1 0 0 \\ 
		\hline                                                     
		13  & 1 1 0 1 & 1 1 & 0 0 0 0 & 0 0 0 0 & 0 0 1 0 & 0 0 1 0 \\ 
		\hline 
	\end{tabular} 
\end{center}

\textbf{C\'odigo:}\\

\textit{Para la m\'aquina de estados:}
\begin{lstlisting}
library ieee;
use ieee.numeric_std.all;
use ieee.std_logic_1164.all;

entity entrada_estado is
	Port (
			reloj: in std_logic;
			entradas: in unsigned(1 downto 0); -- Bit menos significativo representa Si, else Sd
			salidas: out unsigned(3 downto 0)
			);
end entity;

architecture Behavioral of entrada_estado is
signal prueba: unsigned(1 downto 0);
signal sv, sf, lf, lv, salidas_mem, registro_mem, selector_liga, selector_salidas: unsigned(3 downto 0);
signal selector_entradas: std_logic;
signal mem_salida : unsigned(17 downto 0);

begin

-- Instancia de la memoria
memoria: entity work.memoria_ee
	port map(
		direccion => to_integer(registro_mem),
		salidas => mem_salida
	);

-- Multiplexor que elige la entrada
mux_entradas: selector_entradas <= entradas(0) when prueba = "00" else
											  entradas(1) when prueba = "01" else
											  '1';
-- Multiplexor que elige entre liga falsa y verdadera
mux_liga: selector_liga <= lf when selector_entradas = '0' else lv;
-- Multiplexor que elige entre salidas verdaderas o falsas
mux_salidas: selector_salidas <= sf when selector_entradas = '0' else sv;

-- Asignaciones de la salida de la memoria
-- Salidas falsa y verdadera
sf <= mem_salida(3 downto 0);
sv <= mem_salida(7 downto 4);
-- Liga falsa y verdadera
lf <= mem_salida(11 downto 8);
lv <= mem_salida(15 downto 12);
-- Prueba
prueba <= mem_salida(17 downto 16);

-- Registro que direcciona la memoria
reg_mem: process(reloj)
begin
	if rising_edge(reloj) then
		registro_mem <= selector_liga;
	end if;
end process;

-- Registro de las salidas
reg_salida: process(reloj)
begin
	if rising_edge(reloj) then
		salidas <= selector_salidas;
	end if;
end process;

end Behavioral;
\end{lstlisting}

\bigskip
\textit{Para la memoria:}
\begin{lstlisting}

library ieee;
use ieee.std_logic_1164.all;
use ieee.numeric_std.all;

entity memoria_ee is

	generic 
	(
		TAM_PALABRA : natural := 18;
		TAM_MEMORIA : natural := 14
	);

	port 
	(
		direccion	: in natural range 0 to 2**TAM_MEMORIA - 1;
		salidas		: out unsigned((TAM_PALABRA -1) downto 0)
	);

end entity;

architecture rtl of memoria_ee is
	subtype palabra_t is unsigned((TAM_PALABRA-1) downto 0);
	type memoria_t is array(TAM_MEMORIA-1 downto 0) of palabra_t;
	
	signal mem : memoria_t := (
		0 =>  "000001001000000000",
		1 =>  "010101001100000000",
		2 =>  "011100000000001000",
		3 =>  "110100010001000100",
		4 =>  "110000000000010001",
		5 =>  "110110011001000100",
		6 =>  "110111011100100010",
		7 =>  "111000100000100010",
		8 =>  "111001100110001000",
		9 =>  "111010101010001000",
		10 => "111011101100010001",
		11 => "110000000000010001",
		12 => "111101110101000100",
		13 => "110000000000100010"
	);
begin
	salidas <= mem(direccion);
end rtl;

\end{lstlisting}
\bigskip
\textbf{Simulaci\'on:}

\begin{figure}[H]
	\centering
	\includegraphics[width=1.0\textwidth]{input_00_ee}
	\caption{Entradas: 00}
\end{figure}

\begin{figure}[H]
	\centering
	\includegraphics[width=1.0\textwidth]{input_11_ee}
	\caption{Entradas: 11}
\end{figure}

\newpage
\section{Direccionamiento Impl\'icito}

\begin{figure}[H]
	\centering
	\includegraphics[width=0.8\textwidth]{carta_asm_implicito}
	\caption{Carta ASM modificada}
\end{figure}

\noindent \textbf{N\'umero de estados:} 14, por lo que se utilizan 4 bits para representarlos\\
\textbf{Entradas:} $S_{i}$ y $S_{d}$\\
\textbf{Salidas:} Adelante (Ad), Atr\'as (At), Giro Izquierda (GI), Giro Derecha (GD)\\
\textbf{Prueba}:\\
\begin{center}
	\begin{tabular}{|c|c|}
		\hline
		 Entrada & C\'odigo\\
		\hline 
		$S_{i}$ & 0 0 \\ 
		\hline 
		$S_{d}$ & 0 1 \\ 
		\hline 
		$\uparrow Aux$ & 1 1 \\ 
		\hline 
	\end{tabular}
\end{center}

\newpage
\textbf{Carga:}\\
\begin{center}
	\begin{tabular}{|c|c|c|}
		\hline 
		Entrada & VF & Carga \\ 
		\hline 
		0 & 0 & 1 \\ 
		\hline 
		0 & 1 & 0 \\ 
		\hline 
		1 & 0 & 0 \\ 
		\hline 
		1 & 1 & 1 \\ 
		\hline 
	\end{tabular} 
\end{center}

\textbf{Contenido de la memoria:}\\

\begin{center}
	\begin{tabular}{|c|c|c|c|c|c|c|}
		\hline 
		\multicolumn{2}{|c|}{ \multirow{2}{*}{Estado}} & \multirow{2}{*}{Prueba} & \multirow{2}{*}{VF} & \multirow{2}{*}{Liga} & SV & SF \\ 
		\cline{6-7}
		 \multicolumn{2}{|c|}{}& & & &Ad At GI GD & Ad At GI GD \\ 
		\hline 
		0   & 0 0 0 0 & 0 0 & 0 & 1 0 1 1 & 0 0 0 0 & 0 0 0 0\\ 
		\hline                                               
		1   & 0 0 0 1 & 0 1 & 1 & 0 1 0 0 & 0 0 0 0 & 0 0 0 0\\ 
		\hline                                               
		2   & 0 0 1 0 & 1 1 & 1 & 0 0 1 1 & 0 1 0 0 & 0 1 0 0\\ 
		\hline                                               
		3   & 0 0 1 1 & 1 1 & 1 & 0 0 0 0 & 0 0 1 0 & 0 0 1 0\\ 
		\hline                                               
		4   & 0 1 0 0 & 1 1 & 1 & 0 1 0 1 & 0 1 0 0 & 0 1 0 0\\ 
		\hline                                               
		5   & 0 1 0 1 & 1 1 & 1 & 0 1 1 0 & 0 0 1 0 & 0 0 1 0\\ 
		\hline                                               
		6   & 0 1 1 0 & 1 1 & 1 & 0 1 1 1 & 0 0 1 0 & 0 0 1 0\\ 
		\hline                                               
		7   & 0 1 1 1 & 1 1 & 1 & 1 0 0 0 & 1 0 0 0 & 1 0 0 0\\ 
		\hline                                               
		8   & 1 0 0 0 & 1 1 & 1 & 1 0 0 1 & 1 0 0 0 & 1 0 0 0\\ 
		\hline                                               
		9   & 1 0 0 1 & 1 1 & 1 & 1 0 1 0 & 0 0 0 1 & 0 0 0 1\\ 
		\hline                                               
		10  & 1 0 1 0 & 1 1 & 1 & 0 0 0 0 & 0 0 0 1 & 0 0 0 1\\ 
		\hline                                               
		11  & 1 0 1 1 & 0 1 & 0 & 0 0 0 0 & 0 0 0 0 & 1 0 0 0\\ 
		\hline                                               
		12  & 1 1 0 0 & 1 1 & 1 & 1 1 0 1 & 0 1 0 0 & 0 1 0 0\\ 
		\hline                                               
		13  & 1 1 0 1 & 1 1 & 1 & 0 0 0 0 & 0 0 1 0 & 0 0 1 0\\ 
		\hline 
	\end{tabular} 
\end{center}

\textbf{C\'odigo:}

\begin{lstlisting}

library ieee;
  use ieee.std_logic_1164.all;
  use ieee.numeric_std.all;

entity practica6 is
  port (
    clock     : in  std_logic;
    entradas  : in  unsigned(1 downto 0);
    salidas   : out unsigned(3 downto 0)
  );
end entity;

architecture arch of practica6 is
  signal carga_s, vf_s, ent_s: std_logic;
  signal prueba_s: unsigned(1 downto 0);
  signal liga_s, salv_s, salf_s, dir_s: unsigned(3 downto 0);
  signal sal_mem_s: unsigned(14 downto 0);
begin
  cont_e: entity work.contador port map(
    clock => clock,
    data  => liga_s,
    load  => carga_s,
    count => dir_s
  );

  rom_e: entity work.rom port map(
    cs        => '1',
    addr      => dir_s,
    data_out  => sal_mem_s
  );

  reg_outv_p: process(clock) begin
    if rising_edge(clock) then
			salv_s <= sal_mem_s(7 downto 4);
		end if;
	end process;

  reg_outf_p: process(clock) begin
    if rising_edge(clock) then
			salf_s <= sal_mem_s(3 downto 0);
		end if;
	end process;

  prueba_s <= sal_mem_s(14 downto 13);
  vf_s     <= sal_mem_s(12);
  liga_s   <= sal_mem_s(11 downto 8);
  carga_s  <= ent_s xnor vf_S;

  with prueba_s select
    ent_s <=  entradas(0) when  "00",
              entradas(1) when  "01",
              '1'         when  "11",
              '1'         when  others;

  salidas <= salf_s when vf_s = '0' else salv_s;

end architecture;

\end{lstlisting}

\bigskip
\textbf{Simulaci\'on:}

\begin{figure}[H]
	\centering
	\includegraphics[width=1.0\textwidth]{input_00_imp}
	\caption{Entradas: 00}
\end{figure}

\begin{figure}[H]
	\centering
	\includegraphics[width=1.0\textwidth]{input_11_imp}
	\caption{Entradas: 11}
\end{figure}

\section*{Conclusiones}

% Please, put your comments here
\subsection*{\diego}
La tercera práctica se ha vuelto un m\'as interesante ya que se sientan las bases con las que se trabajar\'a a lo largo del semstre. El uso de direccionamientos es ampliamente utilizado por que reduce repetici\'on y creaci\'on de tablas enormes de estados. Entrada-Estado y direccionamiento impl\'icito son buenos acercamientos a lo que m\'as adelante ser\'a el proyecto final que es mucho m\'as cercano a un proyecto real que los ejercicios en clase. Con esto y los constantes ejercicios y tareas se reafirma el entendimiento de los direccionamientos.
\subsection*{\emilio}
\subsection*{\luis}

\end{document}
